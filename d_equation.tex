\documentclass[dvipdfmx, a4paper]{jsarticle}

\usepackage{url}
\usepackage{siunitx}
\usepackage{color}
\usepackage{graphicx}
%\usepackage{tikz}
%\usetikzlibrary{shadows}
\usepackage{comment}
\usepackage{amsmath, amssymb}
\usepackage{theorem}
\usepackage{amsfonts}
\usepackage{bbm, bm}
\usepackage{hyperref}
\usepackage{pxjahyper}
\usepackage{physics}
%\usepackage{float}
%\usepackage{cancel}
\usepackage{tcolorbox}
\tcbuselibrary{breakable, skins, theorems}
\usepackage{listings, jlisting}
\usepackage{here}

\newcommand{\N}{\mathbb{N}}
\newcommand{\Z}{\mathbb{Z}}
\newcommand{\Q}{\mathbb{Q}}
\newcommand{\R}{\mathbb{R}}
\newcommand{\C}{\mathbb{C}}
\newcommand{\x}{\bm{x}}

\lstset{
    language = C++,
    frame = single,
    basicstyle = {\small\ttfamily},
    identifierstyle = {\small},
    commentstyle = {\small\ttfamily \color[rgb]{0.5, 0.5, 0.5}},
    keywordstyle = {\small\ttfamily \color[rgb]{0.7, 0, 0.5}},
    stringstyle = {\small\ttfamily \color[rgb]{0, 0.3, 0.5}}
}

\theorembodyfont{\normalfont}
\newtheorem{theorem}{定理}
\newtheorem{definition}[theorem]{定義}

\title{常微分方程式}
\author{Lei}
\date{\today}

\begin{document}

\maketitle

この資料は, 高橋陽一郎著『微分方程式入門』(基礎数学シリーズ 6 , 東京大学出版会)を参考に、常微分方程式の基本的な内容をまとめ直したものである。

\section{序論}

\subsection{微分方程式とその解}

一般に、未知変数 $x$ のある階数までの導関数 $\frac{d^ix}{dt^i}\ (i=1, \cdots, p)$ の間に与えられた関数関係を $x$ に関する\textgt{常微分方程式}と呼び、関数 $x=x(t)$ が求まればその\textgt{解}であるという. 実 $n$ 空間を $\R^n$ と書く.

\begin{definition}
    $D$ を $\R^{n+1}$ の領域、$f: D\to \R^n$ をベクトル値関数とする. このとき,
    \begin{equation}
        \label{def1}
        \frac{d\x}{dt}=f(t, \x),\ (t, \x)\in D,\ t\in\R,\ \x\in\R^n
    \end{equation}
    の形のものを\textgt{正規形常微分方程式}という。方程式の右辺 $f(t, \x)$ が $t$ によらない関数で与えられるとき、これは\textgt{自励的}であるという.
\end{definition}

\begin{definition}
    $\exists\ I\subset\R$ で定義された $\R^n$ 値関数 $\x(t)=(x_1(t), x_2(t), \cdots, x_n(t))$ が次の3条件を満たすとき、常微分方程式 (\ref{def1}) の解であるという.
    \begin{enumerate}
        \item $\forall t\in I$ に対して $(t, \x(t))\in D$
        \item 関数 $\x(t)$ は微分可能
        \item $\forall t\in I$ に対して, 等式 $\frac{d\x}{dt}(t)=f(t, \x(t))$ が成立
    \end{enumerate}
\end{definition}

幾何学的に考えれば、自励的な常微分方程式の解 $x(t),\ t\in I$ とは

\end{document}