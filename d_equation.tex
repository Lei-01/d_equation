\documentclass[dvipdfmx, a4paper]{jsarticle}

\usepackage{url}
\usepackage{siunitx}
\usepackage{color}
\usepackage{graphicx}
%\usepackage{tikz}
%\usetikzlibrary{shadows}
\usepackage{comment}
\usepackage{amsmath, amssymb}
\usepackage{theorem}
\usepackage{amsfonts}
\usepackage{bbm, bm}
\usepackage{hyperref}
\usepackage{pxjahyper}
\usepackage{physics}
%\usepackage{float}
%\usepackage{cancel}
\usepackage{tcolorbox}
\tcbuselibrary{breakable, skins, theorems}
\usepackage{listings, jlisting}
\usepackage{here}

\newcommand{\N}{\mathbb{N}}
\newcommand{\Z}{\mathbb{Z}}
\newcommand{\Q}{\mathbb{Q}}
\newcommand{\R}{\mathbb{R}}
\newcommand{\C}{\mathbb{C}}
\newcommand{\x}{\bm{x}}

\lstset{
    language = C++,
    frame = single,
    basicstyle = {\small\ttfamily},
    identifierstyle = {\small},
    commentstyle = {\small\ttfamily \color[rgb]{0.5, 0.5, 0.5}},
    keywordstyle = {\small\ttfamily \color[rgb]{0.7, 0, 0.5}},
    stringstyle = {\small\ttfamily \color[rgb]{0, 0.3, 0.5}}
}

\theorembodyfont{\normalfont}
\newtheorem{theorem}{定理}
\newtheorem{definition}[theorem]{定義}
\newtheorem{example}[theorem]{例}

\title{常微分方程式}
\author{Lei}
\date{\today}

\begin{document}

\maketitle

この資料は, 高橋陽一郎著『微分方程式入門』(基礎数学シリーズ 6 , 東京大学出版会)を参考に、常微分方程式の基本的な内容をまとめ直したものである。

\section{序論}

\subsection{微分方程式とその解}

一般に、未知変数 $x$ のある階数までの導関数 $\frac{d^ix}{dt^i}\ (i=1, \cdots, p)$ の間に与えられた関数関係を $x$ に関する\textgt{常微分方程式}と呼び、関数 $x=x(t)$ が求まればその\textgt{解}であるという. 実 $n$ 空間を $\R^n$ と書く.

\begin{definition}
    $D$ を $\R^{n+1}$ の領域、$f: D\to \R^n$ をベクトル値関数とする. このとき,
    \begin{equation}
        \label{def1}
        \frac{d\x}{dt}=f(t, \x),\ (t, \x)\in D,\ t\in\R,\ \x\in\R^n
    \end{equation}
    の形のものを\textgt{正規形常微分方程式}という。方程式の右辺 $f(t, \x)$ が $t$ によらない関数で与えられるとき、これは\textgt{自励的}であるという.
\end{definition}

\begin{definition}
    $\exists\ I\subset\R$ で定義された $\R^n$ 値関数 $\x(t)=(x_1(t), x_2(t), \cdots, x_n(t))$ が次の3条件を満たすとき、常微分方程式 (\ref{def1}) の解であるという.
    \begin{enumerate}
        \item $\forall t\in I$ に対して $(t, \x(t))\in D$
        \item 関数 $\x(t)$ は微分可能
        \item $\forall t\in I$ に対して, 等式 $\frac{d\x}{dt}(t)=f(t, \x(t))$ が成立
    \end{enumerate}
\end{definition}

幾何学的に考えれば、自励的\footnote{方程式 (1) の右辺が $t$ によらない関数で与えられるとき、(1) は自励的という。}な常微分方程式の解 $x(t),\ t\in I$ とは、与えられた領域内にある微分可能な曲線で、各点で与えられているベクトル $f$ を接ベクトルとするもののことである\footnote{解を $\R^n$ 内の曲線と考える時、\textgt{解曲線}ということがある。}。

微分方程式を考察する際、次のような問題が生じてくる。

\begin{itemize}
    \item 運動は定まるのか。言い換えれば
        \begin{enumerate}
            \item (局所)解は存在するのか
            \item 解の定義域はどれだけ広げられるか
            \item 初期値問題\footnote{$D$ の点 $(x_0, t_0)$ を与えたとき、条件\begin{equation}x(t_0)=x_0\end{equation}を満たす解を求めること。}の解は一意的に定まるか
        \end{enumerate}
    \item 更に、次の問題に答えられるか否かは、微分方程式を考えること自体にも関わってくる。
        \begin{enumerate}
            \item 初期値 $x(t_0)=x_0$ に関する解の連続性や微分可能性
            \item 右辺 $f(t, x)$ が更にパラメータに依存するとき、解のパラメータに関する連続性や微分可能性
        \end{enumerate}
\end{itemize}

これらの疑問を考察する前に、いくつか例を見てみることにする。

\begin{example}{(指数関数) }
    $n=1, \alpha\in\R, (t_0, x_0)\in\R^2$ とする。初期値問題
    \[
        \frac{dx}{dt}=\alpha x,\ x(t_0)=x_0\]
    の解は $x(t)=x_0e^{\alpha(t-t_0)}$ であり、定義域は $I=\R$ である。
\end{example}

\begin{example}{(調和振動) }
    $\omega\in\R$ とし、
    \[
        \frac{d^2 u}{dt^2}=-\omega^2 u\]
    を考える。これは正規形ではないが、$x_1=u, x_2=\frac{du}{dt}$ とおけば $x=\left(\begin{matrix}x_1\\x_2\end{matrix}\right)$ に対する正規形方程式
\end{example}

\end{document}